\section{6.1 保留性}%应使用\section{保留性},下同

关于在第 4 章和第 5 章中定义的$\mathrm{E}$ 的保留性理论已通过在转换系统上使用归纳法(规则


5.4)得到证明。%需要引用

定理  6.2 (保留性. {\it 如果} $e$ : $\tau$ {\it 且} $e\mapsto e'$, {\it 那么} $e'$ : $\tau.$

{\it 证明}. 给出两种情况的证明,其余留给读者。考虑规则(5.4b) ,%5.4b应该在括号里面,需要引用,下同

.
$$
\frac{e_{1}\mapsto e_{1}'}{\mathrm{p}\mathrm{l}\mathrm{u}\mathrm{s}(e_{1};e_{2})\mapsto \mathrm{p}1\mathrm{u}\mathrm{s}(e_{1}';e_{2})}
$$
假设 $(e_{1;}\cdot e_{2})$ : $\tau$. 通过类型转化,我们有 $\tau=\mathrm{n}\mathrm{u}\mathrm{m}, e_{1}$ : num, 和 $x.e_{2}$ : num.
%x.e_{2}
归纳可得 e1 : num, 且由此有 plus(e1; $e_{2}$) : num. 连接操作的处理与此相似。
%e1有问题
现在考虑规则(5.4h) ,
$$
\overline{1\mathrm{e}\mathrm{t}(e_{1;}\cdot x.e_{2})\mapsto[e_{1}/x]e_{2}}\ 
$$
假设 let $(e_{1;}.\ x.e2)$ : $\tau_{2}$. 由引理4.2 , 设$e_{1}$ : $\tau_{1}$ , 故有 $x$: $\tau_{1} \vdash x.e_{2}$:
%x.e_{2}
$\tau_{2}$. 通过替换引理4.4中的 $[e_{1}/x]e_{2}$ : $\tau_{2}$, 可以得证。

很容易得到原语操作全是类型保留的。比如, 如果 $a$ nat

且 $b$ nat 且 $a+b=c$, 那么 $c$ nat.
$$
\square 
$$
保留性的证明由转化形式判断上的归纳构成,因为它的论证在于对一个表达式所有可能转化

形式的检测。有时候我们会尝试通过对$e$的结构归纳,或类型归纳来得到证明,但经验告诉


我们,这往往会导致得不到想要的结论,或者根本不能付诸操作。
