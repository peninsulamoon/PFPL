\chapter{求值动态语义}%\chapter{求值动态语义}evaluation dynamics应为求值动态语义,下同
%原译者注:这里evaluation我认为指求值分析,且偏向于分析,故将Evaluation Dynamics翻译成求值分析动态语义,简称成分析动态;经修订者提醒,现在改成求值动态语义更贴切
Chapter 7

求值动态语义

在第5章中,我们用结构化动态语义定义了表达式 $\mathrm{E}$ 的分析。 结构化动态语义对于证明安全性非常有用,%动态结构改为结构化动态语义,下同
但是出于某些目的,比如编写用户手册,另一个被称为{\it 求值动态语义}的公式更可取。求值动态语义是
%求值动态语义是一个表达式和它的值之间的关系
一个表达式与它的值之间的关系,定义时没有详细说明逐步分析的过程。采用成本度量来指定分析的资

源使用情况,用{\it 动态成本}丰富动态分析。一个主要的例子是时间,它是根据表达式的结构化动态语义来计算

表达式所需的转换步骤数。

\subimport{./}{Evaluation_Dynamics}
\subimport{./}{Relating_Structural_and_Evaluation_Dynamics}
\subimport{./}{Type_Safety, Revisited}
\subimport{./}{Cost_Dynamics}
\subimport{./}{notes}
\subimport{./}{exercises}






